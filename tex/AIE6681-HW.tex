\documentclass[
	10pt,
	a4paper,
	% kosection,
	% footnote,
	% nobookmarks,
	% microtype,
	figtabcapt,
%	lwarp
]{oblivoir}

%%%%%%%%%%%%%%%%
% Front Matter %
%%%%%%%%%%%%%%%%

\setcounter{secnumdepth}{5} % section numbering depth


\usepackage[dvipsnames]{xcolor}

\usepackage{fapapersize}
\usefapapersize{*,*,30mm,*,35mm,*}
\definefageometry{default}{30mm,*,35mm,*}[\nopagecolor]
\definefageometry{test}{15mm,25mm,35mm,*}[\pagecolor{cyan!15}]


\usepackage{kotex-logo}

%for reference
\usepackage{hyperref}
\hypersetup{colorlinks=true, linkcolor=black, urlcolor=cyan}
\renewcommand{\figureautorefname}{그림~}
\renewcommand{\tableautorefname}{표~}
% \renewcommand{\sectionautorefname}{foobar}
% \renewcommand{\subsectionautorefname}{foobarbaz}

\usepackage{afterpage}
\usepackage{oblivoir-misc}
\usepackage{graphicx}
\usepackage{fancyvrb}

\usepackage{fontspec-xetex}
% \setmainhangulfont{NanumMyeongjo-YetHangul.ttf}

\usepackage{tabularx}


% for citaiton
% \usepackage{biblatex}
							% Declare packages

%%%%%%%%%%
% Title, Authors, Date %
%%%%%%%%%%

\title{\vspace{-4cm}디지털휴먼엔터테인먼트특론 과제 1}
\author{120220210 고재현}
\date{\today}

\begin{document}

%%%%%%%%%%
% covers %
%%%%%%%%%%


%%%%%%%%%%%%%%%%%%%%%
% Table of Contents %
%%%%%%%%%%%%%%%%%%%%%
\maketitle

% \begin{abstract}
% 이 문서는 2022 인공지능개론(EEE4178) 과목의 설명서이다.

% \end{abstract}

\pagenumbering{roman}                           % Start page numbering in Roman numerals
% \tableofcontents*        						% Add table of contents
% \clearpage

%%%%%%%%%%%%%%%%%
% preliminaries %
%%%%%%%%%%%%%%%%%

\setcounter{table}{0}		                    % Reset table counter
\setcounter{figure}{0}		                    % Reset figure counter

%%%%%%%%%%%%%
% Main Text %
%%%%%%%%%%%%%

\pagenumbering{arabic}							% Start page numbering in Arabic numerals


\section{개요}{\label{sec:intro}}

\subsection{목적}
\begin{itemize}\tightlist
    \item 최근 대두되는 디지털 휴먼 관련 기반 기술들을 조사한다.
    \item 해당 기반 기술을 구현하기 위해 핵심적인 요소들에 대해 고려해 본다.
\end{itemize}

\subsection{요소 기술 및 최근 합성 기술의 동향}
한유아와 같은 디지털 휴먼을 만들기 위해서는 얼굴 합성 기술, 모션 합성 기술, 음성 합성 기술이 필요하다.
사실적인 표현을 위한 캐릭터 모델링 및 렌더링 기술도 중요하지만, 인공지능 기반의 기술들만 서술하도록 한다.
최근 대두되는 생성 모델은 크게 두 가지의 흐름으로 분류할 수 있다.
normalizing flow와 LM과 Transformer를 결합한 모델이 그것이다.
normalizing flow는 그 이름에서 알 수 있듯이, 데이터의 분포를 정규화하는 방식으로 데이터를 생성한다.
학습 속도는 느리지만, inference 속도는 적은 수의 point를 생성하는 데에는 충분하다\cite{TFGMT2022}.
LM 기반의 생성 모델은 고품질의 Tokenizer 및 Transformer 기반 예측 모델의 발전으로
최근 Vall-E\cite{vall-eX}와 같이 빠른 속도의 학습 및 생성이 가능하다.

\subsection{요소 기술별 동향}
\subsubsection{얼굴 합성 기술}
얼굴 합성 기술은 주로 키포인트(랜드마크 등)를 생성한 뒤,
해당 키포인트를 기반으로 모델링을 덮어씌우는 방식으로 수행\cite{MeshTalk2021}되거나,
원본 영상을 변형하여 각 영상 프레임을 생성하는 방식\cite{MakeItTalk2020}으로 수행된다.
스마일게이트 AI Media Team의 블로그 자료를 보면,
이와 관련하여 원본 영상을 변형하여 각 영상 프레임을 생성하는 방식의 실시간 처리를 시도한 바가 있다.
해당 방식은

\section{디지털 휴먼 모델의 발전 방향}

 

%%%%%%%%%%%%%%%%%%%%%%
% References Section %
%%%%%%%%%%%%%%%%%%%%%%
\bibliographystyle{ieeetr}						% Declare bibliography style
\bibliography{ref.bib}							% Declare bibliography file

\clearpage

\end{document}
