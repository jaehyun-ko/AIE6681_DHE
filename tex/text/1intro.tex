\section{개요}{\label{sec:intro}}

\subsection{목적}
\begin{itemize}\tightlist
    \item 최근 대두되는 디지털 휴먼 관련 기반 기술들을 조사한다.
    \item 해당 기반 기술을 구현하기 위해 핵심적인 요소들에 대해 고려해 본다.
\end{itemize}

\subsection{요소 기술}
한유아와 같은 디지털 휴먼을 만들기 위해서는 얼굴 합성 기술, 모션 합성 기술, 음성 합성 기술이 필요하다.
사실적인 표현을 위한 캐릭터 모델링 및 렌더링 기술도 중요하지만, 인공지능 기반의 기술들만 서술하도록 한다.
얼굴 합성 및 모션 합성 기술은 주로 키포인트(랜드마크 등)를 생성한 뒤, 해당 키포인트를 기반으로 모델링을 덮어씌우는 방식으로 수행\cite{MeshTalk2021}되거나,
원본 영상을 변형하여 각 영상 프레임을 생성하는 방식으로 수행된다.

\subsection{선정 기술 및 논문}
